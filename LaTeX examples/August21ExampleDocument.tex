\documentclass{article}
\usepackage[utf8]{inputenc}
\usepackage[margin=1in]{geometry}

\title{Math 340 Day 2}
\author{Zane and Dr. McNelis}
\date{21 August, 2019}

\begin{document}

\maketitle

\section{Lists}
% How to make an ordered/numbered list
This list contains 5 items, but this is not one of them.
\begin{enumerate}
    \item Bears
    \item Beets
    \begin{enumerate}
        \item Roasted
        \item Raw
        \item Pickled
    \end{enumerate}
    \item Battlestar Galactica
    \item Fourth list item
    \item Fifth list item
\end{enumerate}

\vspace{0.5cm} % Remember, this makes a block of vertical space on the page.

% How to make an unordered list / bullet points
Types of bears. (Your list is about your favorite TV shows!)
\begin{itemize}
    \item Polar bears.
    \item Black bears.
    \begin{itemize}
        \item No shoulder hump.
        \item Long, pointed ears.
        \item Flat facial profile.
    \end{itemize}
    \item Grizzly bears.
    \begin{itemize}
        \item Characteristic shoulder hump.
        \item Small, round ears.
        \item Concave facial profile.
    \end{itemize}
\end{itemize}

\section{The Tabular Environment}

\begin{tabular}{ccccc} % placeholders specify alignment! c, l, or r works. This table has 5 centered columns. Remember & goes to next column, \\ goes to next row. You can add | between placeholders to make vertical lines in your table. E.g. try |c|c|c|c|c| to see.
M & T & W & R & F \\
\hline \\ % This prints a horizontal line.
MATH 340 & & MATH 340 & & MATH 340 \\
ECET 321 & & ECET 321 & & ECET 321 \\
 & MATH 370 & & MATH 370 &  \\
\end{tabular}


\end{document}
