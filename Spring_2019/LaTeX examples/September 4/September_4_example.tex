\documentclass{article}

\usepackage{url} % Since we included a \url{} in our BibTeX file, we need to have the url package here.

% Today in class we covered: how to make a bibliography

\title{MATH340\_Day7}
\author{Zane and Dr. McNelis}
\date{4 September, 2019}

\begin{document}
% How to include something in your bibliography without an in-text citation
\nocite{*} % the * means all, or you can specifiy nicknames.

\maketitle

\section{Bibliographies}
% You will need to make a second file in the same project to get a bibliography. The bibliography will be inside of a .bib file (short for BibTeX). 

% The \cite{} command references something from your bibliography, adds an in-text reference, and automatically adds the citation to your bibliography wherever you put it.

This sentence needs to have a citation~\cite{fakebook}. % The tilde makes a "sticky space" that can't go away during formatting. It is optional.

RNA inference is a fun way to figure stuff out in genetics. These people used it to look at the \textit{C. elegans} genome \cite{kamath2003systematic}.

% How to insert your bibliography into this document:
\bibliographystyle{siam} % There are a lot of styles, you can find one online for most styles of citations.
\bibliography{example} % Note that you do not need to add the .bib file extension here because the compiler knows to look for a .bib file.

% If you are NOT on overleaf (e.g. using a distribution on your machine), you need to compile four times: LaTeX, BibTex, LaTeX, LaTeX. Overleaf does all this in the background for you.

\end{document}

% An example citation:
% @type_of_course{nickname,
% field_one = {},
% field_two = {},
% etc. = {} }

% We got this citation from google scholar!